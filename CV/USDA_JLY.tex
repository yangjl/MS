\documentclass[letterpaper]{article}

\usepackage{hyperref}
\usepackage{geometry}
\usepackage{etaremune}
%\usepackage{eurofont}
\usepackage{verbatim}

% Comment the following lines to use the default Computer Modern font
% instead of the Palatino font provided by the mathpazo package.
% Remove the 'osf' bit if you don't like the old style figures.
\usepackage[T1]{fontenc}
\usepackage[sc,osf]{mathpazo}
\usepackage{setspace}
% Set your name here
\def\name{Jinliang Yang}

% Replace this with a link to your CV if you like, or set it empty
% (as in \def\footerlink{}) to remove the link in the footer:
\def\footerlink{}

%Add in-line comments
\newcommand{\ignore}[1]{}

% The following metadata will show up in the PDF properties
\hypersetup{
  colorlinks = true,
  urlcolor = blue,
  pdfauthor = {\name},
  pdfkeywords = {quantitative genetics, maize, GWAS},
  pdftitle = {\name: Curriculum Vitae},
  pdfsubject = {Curriculum Vitae},
  pdfpagemode = UseNone
}

\geometry{
  body={6.5in, 9in},
  left=1.0in,
  top=1.0in
}

% Customize page headers
\pagestyle{myheadings}
%\markright{\name}
\thispagestyle{empty}

% Custom section fonts
\usepackage{sectsty}
\sectionfont{\rmfamily\mdseries\Large}
\subsectionfont{\rmfamily\mdseries\itshape\large}


% Other possible font commands include:
% \ttfamily for teletype,
% \sffamily for sans serif,
% \bfseries for bold,
% \scshape for small caps,
% \normalsize, \large, \Large, \LARGE sizes.

% Don't indent paragraphs.
\setlength\parindent{0em}

% Make lists without bullets
\renewenvironment{itemize}{
  \begin{list}{}{
    \setlength{\leftmargin}{1.5em}
  }
}{
  \end{list}
}

\begin{document}

% Place name at left
%{\huge \name}
% Alternatively, print name centered and bold:
\centerline{\huge \bf \name}

\vspace{0.25in}

%ADDRESS
%%% address and contacts
\begin{minipage}{0.55\linewidth}
  \href{http://www.plantsciences.ucdavis.edu/plantsciences/}{Department of Plant Sciences}\\
  \href{http://www.ucdavis.edu/}{University of California} \\
  One Shields Ave.\\
  Davis, CA 95616 \\
\end{minipage}
\begin{minipage}{0.35\linewidth}
  \begin{tabular}{ll}
    Phone: & (515) 509-4552 \\
    Email: & \href{mailto:jolyang@ucdavis.edu}{jolyang@ucdavis.edu} \\
    Web: & \href{http://yangjl.com}{http://yangjl.com} \\
  \end{tabular}
\end{minipage}


%EDUCATION
\section*{Education}
\begin{minipage}{0.2\linewidth}
  2008-2014 \\
\end{minipage}
\begin{minipage}{0.7\linewidth}
  PhD Interdepartmental Genetics, Iowa State University, Ames, IA, US \\
\end{minipage}

\begin{minipage}{0.2\linewidth}
  2005-2008 \\
\end{minipage}
\begin{minipage}{0.7\linewidth}
  MS Biochemistry and Molecular Biology, China Agricultural University, Beijing, CHN \\
\end{minipage}

\begin{minipage}{0.2\linewidth}
  2001-2005 \\
\end{minipage}
\begin{minipage}{0.7\linewidth}
  BS in Bioengineering, China Agricultural University, Beijing, CHN \\
\end{minipage}

%EMPLOYMENT
\section*{Professional Experience}
\begin{minipage}{0.2\linewidth}
  2014-present \\
\end{minipage}
\begin{minipage}{0.7\linewidth}
  Postdoctoral Researcher with Jeffrey Ross-Ibarra, University of California Davis \\
\end{minipage}

\begin{minipage}{0.2\linewidth}
  2008-2014 \\
\end{minipage}
\begin{minipage}{0.7\linewidth}
  Research Assitant with Patrick Schnable, Iowa State University \\
\end{minipage}

\begin{minipage}{0.2\linewidth}
  2006-2008 \\
\end{minipage}
\begin{minipage}{0.7\linewidth}
  Research Assitant with Jinsheng Lai, China Agricultural University \\
\end{minipage}


%Teching Experience
\section*{Teaching Experience}
\begin {itemize}
\item Guest lecture, UC Davis, Ecological Genomics (ECS243, graduate) (2015)
%\item University of California MEXUS Travel Grant 1999
\end{itemize}

%Awards
\section*{Selected Awards}
\begin{itemize}
\item Sui Tong Chan Fung Fund for the Promotion of Study and Research in Genetics (Nov. 2013)
%\item W. Young and W.E. Loomis travel award (Feb. 2012)
\item W. Young and W.E. Loomis travel award (Oct. 2012)
\item PhD Scholarship from Chinese Scholarship Council (2008-2012)
%\item Scholarship for outstanding learning, China Agricultural University (2002)

\end{itemize}
%\end{comment}

%SEMINARS
\section*{Invited Presentation}
\begin{itemize}
\item October 2014 Plant Sciences Departmental Seminar, UC Daivs, CA, Using next-generation sequencing for genome-wide association and prediction in plants.

\item March 2014 Huazhong Agricultural University, Wuhan, China, Big Data meets Genetics: GWAS and Genomic Selection of Yield Related Traits in Maize.
\item March 2014 56th Annual Maize Genetics Conference, Bejing, China, Insights into Heterosis.

\item Feb. 2013 Gordon Research Conference on Quantitative Genetics and Genomics, Galveston, TX , GBS-enabled GWAS: Identification and Validation of Kernel Row Number-associated SNPs in Maize.

\item Feb. 2012 AB\&G Seminar Series, Ames, IA, Identification and validation of maize loci controlling a yield component trait via 2nd generation Bayesian-based GWAS.
\end{itemize}

%PUBS
\section*{Publications} 
% {\small(lab members in bold, $^*$equal contribution, $^\dagger$cover article, $^\ddagger$undergraduate, $^\S$corresponding)}
%SUBMITTED
\begin{itemize}

\item {\bf Yang J}, Yeh CT, Fernando RL, Dekkers JCM, Garrick DJ, Nettleton D, Schnable PS. Identification and Genetic Validation of Nucleotide Variants Associated with the Kernel Row Number Trait of Maize: an Empirical Comparison of GWAS Approaches. (\textsc{submitted})

\item {\bf Yang J}, Jiang H, Yeh CT, Yu J, Jeddeloh JA, Nettleton D, and Schnable PS. Identification of SNPs controlling maize Kernel Row Number (KRN) via an Extreme Phenotype Genome-Wide Association Study (XP-GWAS). (\textsc{submitted})
%CITES:10457251071989199418

\item {\bf Yang J}, Li L, Jiang H, Yeh CT, Fernando RL, Dekkers JCM, Garrick DJ, Nettleton D, and Schnable PS. Dominant Gene Action Accounts for much of the Missing Heritability in a GWAS and Provides Insight into Heterosis. (\textsc{in preparation}) 

\item Liu S, Ying K, Yeh CT, {\bf Yang J}, Swanson-Wagner RA, Wu W, Richmond T, Gerhardt DJ, Lai J, Springer N, Nettleton D, Jeddeloh JA, and Schnable PS (2012). Changes in genome content generated via segregation of non-allelic homologs. The Plant journal: for cell and molecular biology, 72(3), 390-9. 

\item Koesterke L, Stanzione D, Vaughn M, Welch SM, Kusnierczyk W, {\bf Yang J}, Yeh CT, Nettleton D, and Schnable PS (2011). An Efficient and Scalable Implementation of SNP-Pair Interaction Testing for Genetic Association Studies. In Proceedings of the 2011 IEEE International Symposium on Parallel and Distributed Processing Workshops and PhD Forum (IPDPSW '11). IEEE Computer Society, Washington, DC, USA, 523-530.

\end{itemize}

%PRINT/PRESS
\section*{Patent Application}
\begin{itemize}
\item  Schnable PS, {\bf Yang J}, Swanson-Wagner RA, Nettleton D (2011). QTL regulating ear productivity traits in maize. U.S. Patent No. 8779233. Filed July 12, 2011. 

\item Schnable PS, {\bf Yang J} (2013), Identification of QTLs and trait-associated SNPs controlling six yield component traits in maize. (\textsc{pending})
\item Schnable PS, OTT A , {\bf Yang J} (2014), Intercrossed ex-PVP lines. (\textsc{pending})

\end{itemize}



\end{document}
