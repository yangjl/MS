\documentclass[11pt]{article}
\usepackage{times}
\usepackage{ucs}
\usepackage[utf8x]{inputenc} 
\usepackage{url}
\usepackage{etaremune}
\usepackage[top=2.54cm, bottom=2.54cm, left=2.54cm, right=2.54cm]{geometry}
\usepackage{hyperref}
\usepackage{setspace} 
 
\setstretch{1} 
 
\renewcommand{\thesection}{(\alph{section})}
\newcommand{\ignore}[1]{}
 
\begin{document}
 
\begin{center}
    \begin{Large}
		\sf\textbf{\uppercase{jinliang Yang}}
	\end{Large}
\end{center}

%%% address and contacts
\begin{minipage}{0.55\linewidth}
  \href{http://www.plantsciences.ucdavis.edu/plantsciences/}{Department of Plant Sciences}\\
  \href{http://www.ucdavis.edu/}{University of California} \\
  One Shields Ave.\\
  Davis, CA 95616
\end{minipage}

\begin{minipage}{0.35\linewidth}
  \begin{tabular}{ll}
    Phone: & (515) 509-4552 \\
    Email: & \href{mailto:jolyang@ucdavis.edu}{jolyang@ucdavis.edu} \\
    Web: & \href{http://yangjl.com}{http://yangjl.com} \\
  \end{tabular}
\end{minipage}
\bigskip
 
%\section{Expertise}
 
%Dr.\ Ross-Ibarra's research focuses on the population genetics of maize and its wild relatives.  His group uses population genetics to investigate the evolution of natural and cultivated populations, identify loci important for adaptation, and understand the evolutionary forces shaping diversity in the maize genome.    
%\bigskip
 
\section*{Education}
 
\begin{tabular}{l l l l}
Institution    \hspace{52mm}              &   Area  \hspace{10mm}     & Degree / Training  \hspace{13mm}    & Dates \\
\hline
University of California Riverside & Botany & BA & 1998 \\
University of California Riverside & Botany & MS & 2000 \\
University of Georgia & Genetics & PhD & 2006\\
University of California Irvine & Genetics & Postdoctoral Research & 2006--2008 \\
\hline 
\end{tabular}
 
\section*{Professional Appointments}
 
\begin{tabular}{l l l}
Institution                              & Position                 & Dates\\
\hline
Associate Professor & University of California Davis &		2012-present \\
Assistant Professor & University of California Davis &		2009-2012 \\
Profesor de Asignatura & Universidad Nacional Aut\'{o}noma de M\'{e}xico & 2001 \\
\hline
\end{tabular}
 
\section*{Synergistic Activities}
 
\begin{itemize} \setlength{\itemsep}{0pt} \setlength{\parskip}{2pt} \setlength{\parsep}{0pt}
 
\item Co-PI on the NSF-funded Biology of Rare Alleles, exploring functional diversity of rare alleles in maize and methods to utilize these for crop improvment.
 
\item DuPont Young Professor, 2012-2015
 
\item Faculty advisor, Pioneer Hi-Bred graduate student symposium in plant breeding, 2012-2015
 
%\item Scientific Advisory Board, AMAIZING Project, 2011-present 
 
\item Lead PI, NSF-funded US-Mexico student exchange program, 2011-present
 
\item Presidential Early Career Award for Scientists and Engineers 2009
 
\end{itemize}
 
\section*{Former advisees}
\begin{itemize} \setlength{\itemsep}{0pt} \setlength{\parskip}{2pt} \setlength{\parsep}{0pt}
\item Joost van Heerwaarden, Research Scientist, University of Wageningen
\item Matthew Hufford, Asst. Professor, Iowa State University
\item Sofiane Mezmouk, Research Scientist, KWS SAAT
\item Shohei Takuno, Asst. Professor, SOKENDAI
\item Tanja Pyh\"aj\"arvi, Asst. Professor, University Oulu
\end{itemize}
 
\section*{Publications}
\begin{etaremune}
%\item Hake S, {\bf Ross-Ibarra J}. Genetic, evolutionary and plant breeding insights from the domestication of maize. \tetsc{eLife} \emph{In Press}
 
\item Fonseca RR, Smith B, Wales N, Cappellini E, Skoglund P, Fumagalli M, Samaniego JA, Caroe C, Avila-Arcos MC, Hufnagel D, Korneliussen TS, Vieira FG, Jakobsson M, Arriaza B, Willerslev E, Nielsen R, Hufford MB, Albrechtsen A,  {\bf Ross-Ibarra J}, Gilbert MT (2015) The origin and evolution of maize in the American Southwest. \textsc{Nature Plants} 1(1)
%CITES:14394472478509816248,10810382578079898537
 
\item Dyer GA, L\'opez-Feldman A, Y\'unez-Naude A, Taylor JE, {\bf Ross-Ibarra J} (2015) Reply to Brush \emph{et al.}: A wake up call for crop conservation science. PNAS 112 (1), E2-E2 (letter).
%CITES:0
 
\item Makarevitch I, Waters M, West P, {\bf Stitzer M}, {\bf Ross-Ibarra, J}, Springer NM (2015) Mobile elements contribute to activation of genes in response to abiotic stress. \textsc{PLoS Genetics} 11 (1): e1004915. %Preprint: \url{http://biorxiv.org/content/early/2014/08/15/008052}
%CITES:0
 
\item Tiffin P, {\bf Ross-Ibarra J}. Advances and limits of using population genetics to understand local adaptation. \textsc{Trends in Ecology and Evolution} 29:673-680 %Preprint: \url{https://peerj.com/preprints/488/}
%CITES:2471984348452499818
 
\item {\bf Bilinski P}, {\bf Distor KD}, {\bf Gutierez-Lopez J}, {\bf Mendoza Mendoza G}, Shi J, Dawe K,  {\bf Ross-Ibarra J}$^\S$ (2014) Diversity and evolution of centromere repeats in the maize genome. \textsc{Chromosoma} 0009-5915
%CITES:4504435895012493115
%\\ Preprint: http://biorxiv.org/content/early/2014/05/12/005058
 
\item {\bf Mezmouk S}, {\bf Ross-Ibarra J}$^\S$ (2014) The pattern and distribution of deleterious mutations in maize. (2014) \textsc{G3} 4:163-171
%CITES:0
%Preprint: http://arxiv.org/abs/1308.0380
 
\item Waters AJ, {\bf Bilinski P}, Eichten SR, Vaughn MW, {\bf Ross-Ibarra J}, Gehring M, Springer NM (2013) Comprehensive analysis of imprinted genes in maize reveals allelic variation for imprinting and limited conservation with other species. \textsc{PNAS} 110:19639-19644 
%Preprint: http://arxiv.org/abs/1307.7678
%CITES:16494053027693693141
 
\item {\bf Pyh\"aj\"arvi T}, {\bf Hufford MB}, {\bf Mezmouk S}, {\bf Ross-Ibarra J}$^\S$ (2013) Complex patterns of local adaptation in teosinte. \textsc{Genome Biology and Evolution} 5: 1594-1609.$^\dagger$
%CITES:4348910575877017766
%Preprint: \emph{http://arxiv.org/abs/1208.0634}
 
\item Wills DM, Whipple C, {\bf Takuno S}, Kursel LE, Shannon LM, {\bf Ross-Ibarra J}, Doebley JF (2013) From many, one: genetic control of prolificacy during maize domestication. \textsc{PLoS Genetics} 9(6): e1003604. %Preprint: \emph{http://arxiv.org/abs/1303.0882}
%CITES:6107325531042162678
 
\item McCouch S, Baute GJ, Bradeen J, Bramel P, Bretting PK, Buckler E, Burke JM, Charest D, Cloutier S, Cole G, Dempewolf H, Dingkuhn M, Feuillet C, Gepts, P, Grattapaglia D, Guarino L, Jackson S, Knapp S, Langridge P, Lawton-Rauh A, Lijua Q, Lusty C, Michael T, Myles S, Naito K, Nelson RL, Pontarollo R, Richards CM, Rieseberg L, {\bf Ross-Ibarra J}, Rounsley S, Hamilton RS, Schurr U, Stein N, Tomooka N, van der Knaap E, van Tassel D, Toll J, Valls J, Varshney RK, Ward J, Waugh R, Wenzl P, Zamir. (2013) Agriculture: Feeding the future. \textsc{Nature} 499:23-24
%CITES:4910849057817973933
 
\item {\bf Hufford MB}, Lubinsky P, {\bf Pyh\"aj\"arvi T}, {\bf Devengenzo MT}$^\ddagger$, Ellstrand NC, {\bf Ross-Ibarra J}$^\S$ (2013) The genomic signature of crop-wild introgression in maize. \textsc{PLoS Genetics} 9(5): e1003477. %Preprint: \emph{http://arxiv.org/abs/1208.3894}
%CITES:16496318242013452561
 
\item {\bf Provance MC}$^\S$, Garcia Ruiz I, {\bf Thommes C}$^\ddagger$, {\bf Ross-Ibarra J} (2013) Population genetics and ethnobotany of cultivated \emph{Diospyros riojae} G\'omez Pompa (Ebenaceae), an endangered fruit crop from Mexico. \textsc{Genetic Resources and Crop Evolution} 60: 2171-2182.
%Preprint: \emph{http://dx.doi.org/10.6084/m9.figshare.105841}
%CITES:15526564212076251883
 
\item Melters DP$^*$, Bradnam KR$^*$, Young HA, Telis N, May MR, Ruby JG, Sebra R, Peluso P, Eid J, Rank D, Fernando Garcia J, DeRisi J, Smith T, Tobias C, {\bf Ross-Ibarra J}$^\S$, Korf IF$^\S$, Chan SW-L. (2013) Patterns of centromere tandem repeat evolution in 282 animal and plant genomes. \textsc{Genome Biology} 14:R10 
%Preprint: \emph{http://arxiv.org/abs/1209.4967}
%CITES:4643344870051815303
 
\item Kanizay LB, {\bf Pyh\"aj\"arvi T}, Lowry E, {\bf Hufford MB}, Peterson DG, {\bf Ross-Ibarra J}, Dawe RK (2013) Diversity and abundance of the Abnormal chromosome 10 meiotic drive complex in \emph{Zea mays}. \textsc{Heredity} 110: 570-577.
%CITES:9054275167516914311
 
\item {\bf Hufford MB}, {\bf Bilinski P}, {\bf Pyh\"aj\"arvi T}, {\bf Ross-Ibarra J}$^\S$ (2012) Teosinte as a model system for population and ecological genomics. \textsc{Trends in Genetics} 12:606-615$^\dagger$
%CITES:17241763016461383750
 
\item Mu\~{n}oz Diez C, Vitte C, {\bf Ross-Ibarra J}, Gaut BS, Tenaillon MI (2012) Using nextgen sequencing to investigate genome size variation and transposable element content. \emph{In} Grandbastien M-A, Casacuberta JM, editors. \textsc{Topics in Current Genetics} v24: Plant Transposable Elements - Impact on Genome Structure \& Function. pp. 41-58 
%CITES:5019892223730894925
 
\item  {\bf van Heerwaarden J}$^\S$, {\bf Hufford MB}, {\bf Ross-Ibarra J}$^\S$ (2012) Historical genomics of North American maize. \textsc{PNAS} 109: 12420-12425
%CITES:3745727334869718431
 
\item Swanson-Wagner R, Briskine R, Schaefer R, {\bf Hufford MB}, {\bf Ross-Ibarra J}, Myers CL, Tiffin P, Springer NM.  Reshaping of the maize transcriptome by domestication. (2012) \textsc{PNAS}  109: 11878-11883
%CITES:935006617287790715
 
\item {\bf Hufford MB}$^*$, Xun X$^*$, {\bf van Heerwaarden J}$^*$, {\bf Pyh\"aj\"arvi T}$^*$, Chia J-M, Cartwright RA, Elshire RJ, Glaubitz JC, Guill KE, Kaeppler S, Lai J, Morrell PL, Shannon LM, Song C, Spinger NM, Swanson-Wagner RA, Tiffin P, Wang J, Zhang G, Doebley J, McMullen MD, Ware D, Buckler ES$^\S$, Yang S$^\S$, {\bf Ross-Ibarra J}$^\S$ (2012) Comparative population genomics of maize domestication and improvement. \textsc{Nature Genetics} 44:808-811$^\dagger$
%CITES:2254239507517002236
 
\item  Chia J-M$^*$, Song C$^*$, Bradbury P, Costich D, de Leon N, Doebley JC, Elshire RJ, Gaut BS, Geller L, Glaubitz JC, Gore M, Guill KE, Holland J,  {\bf Hufford MB}, Lai J, Li M, Liu X, Lu Y, McCombie R, Nelson R, Poland J, Prasanna BM,  {\bf Pyh\"aj\"arvi T}, Rong T, Sekhon RS,  Sun Q, Tenaillon M, Tian F, Wang J, Xu X, Zhang Z, Kaeppler S, {\bf Ross-Ibarra J}, McMullen M, Buckler ES, Zhang G, Xu Y, Ware, D (2012) Maize HapMap2 identifies extant variation from a genome in flux. \textsc{Nature Genetics} 44:803-807$^\dagger$
%CITES:2228837645322681220
 
\item Fang Z, {\bf Pyh\"aj\"arvi T}, Weber AL, Dawe RK, Glaubitz JC, S\'{a}nchez Gonz\'{a}lez J, {\bf Ross-Ibarra C}, Doebley J, Morrell PL$^\S$, {\bf Ross-Ibarra J}$^\S$  (2012) Megabase-scale inversion polymorphism in the wild ancestor of maize. \textsc{Genetics} 191:883-894 
%CITES:6473644606884467791
 
\item Cook JP, McMullen MD, Holland JB, Tian F, Bradbury P, {\bf Ross-Ibarra J}, Buckler ES, Flint-Garcia SA (2012) Genetic architecture of maize kernel composition in the Nested Association Mapping and Inbred Association panels.  \textsc{Plant Physiology} 158: 824-834
%CITES:11097878024256170139
 
\item Morrell PL, Buckler ES, {\bf Ross-Ibarra J}$^\S$ (2012) Crop genomics: advances and applications.  \textsc{Nature Reviews Genetics} 13:85-96$^\dagger$
%CITES:10635446355677669887
 
\item Studer A, Zhao Q, {\bf Ross-Ibarra J}, Doebley J (2011) Identification of a functional transposon insertion in the maize domestication gene \emph{tb1}.  \textsc{Nature Genetics} 43:1160-1163.
%CITES:11888569066379603503
 
\item {\bf van Heerwaarden J}$^\S$, Doebley J, Briggs WH, Glaubitz JC, Goodman MM, S\'{a}nchez Gonz\'{a}lez JJ, {\bf Ross-Ibarra J}$^\S$ (2011) Genetic signals of origin, spread and introgression in a large sample of maize landraces. PNAS 108: 1088-1092
%CITES:11944729109494726624
 
\item {\bf Hufford MB}$^\S$, Gepts P, {\bf Ross-Ibarra J} (2011) Influence of cryptic population structure on observed mating patterns in the wild progenitor of maize (\emph{Zea mays} ssp. \emph{parviglumis}).  \textsc{Molecular Ecology} 20: 46-55
%CITES:13621041706202411296
 
\item Tenaillon MI, {\bf Hufford MB}, Gaut BS, {\bf Ross-Ibarra J}$^\S$ (2011)  Genome size and TE content as determined by high-throughput sequencing in maize and \emph{Zea luxurians}.  \textsc{Genome Biology and Evolution } 3: 219-229
%CITES:8095218463542481067
 
\item Eckert AJ, {\bf van Heerwaarden J}, Wegrzyn JL, Nelson CD, {\bf Ross-Ibarra J}, Gonz\'{a}lez-Mart\'{i}nez SC, and Neale DB (2010) Patterns of population structure and environmental associations to aridity across the range of loblolly pine (\emph{Pinus taeda} L, Pinaceae).  \textsc{Genetics} 185: 969-982
%CITES:8457237997922496538
 
\item Fuchs EJ, {\bf Ross-Ibarra J}$^\S$, Barrantes G (2010) Reproductive biology of \emph{Macleania rupestris}: a pollen-limited Neotropical cloud-forest species in Costa Rica.  \textsc{Journal of Tropical Ecology} 26: 351-354
%CITES:14583174141795727444
 
\item Whitney KD, Baack EJ, Hamrick JL, Godt MJW, Barringer BC, Bennett MD, Eckert CG, Goodwillie C, Kalisz S, Leitch I, {\bf Ross-Ibarra J} (2010) A role for nonadaptive processes in plant genome size evolution?  \textsc{Evolution} 64: 2097-2109
%CITES:17559374760333968488
 
\item {\bf van Heerwaarden J}, {\bf Ross-Ibarra J}$^\S$, Doebley J, Glaubitz JC, S\'{a}nchez Gonz\'{a}lez J, Gaut BS, Eguiarte LE (2010) Fine scale genetic structure in the wild ancestor of maize (\emph{Zea mays} ssp. \emph{parviglumis}).  \textsc{Molecular Ecology} 19: 1162-1173
%CITES:9340353919503578071
 
\item Shi J, Wolf S, Burke J, Presting G, {\bf Ross-Ibarra J}, Dawe RK (2010) High frequency gene conversion in centromere cores.  \textsc{PLoS Biology} 8: e1000327
%CITES:15554747939087267209
 
\item Hollister JD, {\bf Ross-Ibarra J}, Gaut BS (2010) Indel-associated mutation rate varies with mating system in flowering plants.  \textsc{Molecular Biology and Evolution} 27: 409-416.
%CITES:1965523728393558029
 
\item {\bf van Heerwaarden J}, van Eeuwijk FA, {\bf Ross-Ibarra J} (2010) Genetic diversity in a crop metapopulation.  \textsc{Heredity} 104: 28-39
%CITES:14838883186010671400
 
\item Gore MA$^*$, Chia JM$^*$, Elshire RJ, Sun Q, Ersoz ES, Hurwitz BL, Peiffer JA, McMullen MD, Grills GS, {\bf Ross-Ibarra J}, Ware DH, Buckler ES (2009) A first-generation haplotype map of maize.  \textsc{Science 326}: 1115-1117.
%CITES:1250430020833640405
 
\item {\bf May MR}$^\ddagger$, {\bf Provance MC}, Sanders AC, Ellstrand NC, {\bf Ross-Ibarra J}$^\S$ (2009) A pleistocene clone of Palmer's Oak persisting in Southern California.  \textsc{PLoS ONE} 4: e8346.
%CITES:15690173176823572703
 
\item Zhang LB, Zhu Q, Wu ZQ, {\bf Ross-Ibarra J}, Gaut BS, Ge S, Sang T (2009) Selection on grain shattering genes and rates of rice domestication.  \textsc{New Phytologist} 184: 708-720.
%CITES:853064656161512620
 
\item {\bf Ross-Ibarra J}, Tenaillon M, Gaut BS (2009) Historical divergence and gene flow in the genus Zea.  \textsc{Genetics} 181: 1399-1413.
%CITES:12731906589694221305
 
\item {\bf Ross-Ibarra J}$^*$, Wright SI$^*$, Foxe JP, Kawabe A, DeRose-Wilson L, Gos G, Charlesworth D, Gaut BS (2008) Patterns of polymorphism and demographic history in natural populations of \emph{Arabidopsis lyrata}.  \textsc{PLoS ONE} 3: e2411.
%CITES:12619209345184003027
 
\item Lockton S, {\bf Ross-Ibarra J}, Gaut BS (2008) Demography and weak selection drive patterns of transposable element diversity in natural populations of \emph{Arabidopsis lyrata}. PNAS 105: 13965-13970.
%CITES:15151328051935459123
 
\item {\bf Ross-Ibarra J}$^\S$, Gaut BS (2008) Multiple domestications do not appear monophyletic. PNAS 105: E105 (letter).
%CITES:14659039718259909589
 
\item Gaut BS, {\bf Ross-Ibarra J} (2008) Selection on major components of angiosperm genomes.  \textsc{Science} 320: 484-486.
%CITES:10471445648121255184
 
\item {\bf Ross-Ibarra J}, Morrell PL, Gaut BS (2007) Plant domestication, a unique opportunity to identify the genetic basis of adaptation. PNAS 104 Suppl 1: 8641-8648. 
%CITES:4061357872113450280
 
\item {\bf Ross-Ibarra J}$^\S$ (2007) Genome size and recombination in angiosperms: a second look.  \textsc{Journal of Evolutionary Biology} 20: 800-806.
%CITES:684436662966421061
 
\item Wares JP, Barber PH, {\bf Ross-Ibarra J}, Sotka EE, Toonen RJ (2006) Mitochondrial DNA and population size.  \textsc{Science} 314: 1388-90 (letter).
%CITES:7029521719248710591
 
\item {\bf Ross-Ibarra J}$^\S$ (2005) QTL and the study of plant domestication.  \textsc{Genetica} 123: 197-204. 
%CITES:7426372046771115962
 
\item {\bf Ross-Ibarra J}$^\S$ (2004) The evolution of recombination under domestication: a test of two hypotheses.  \textsc{American Naturalist} 163: 105-112.
%CITES:3095269849835314095
 
\item {\bf Ross-Ibarra J} (2003) Origin and domestication of chaya (\emph{Cnidoscolus aconitifolius} Mill I. M. Johnst): Mayan spinach.  \textsc{Mexican Studies} 19: 287-302.
%CITES:4811395532917369896
 
\item {\bf Ross-Ibarra J}$^\S$, Molina-Cruz A (2002) The ethnobotany of Chaya (\emph{Cnidoscolus aconitifolius} ssp. \emph{aconitifolius} Breckon): A nutritious Maya vegetable.  \textsc{Economic Botany} 56: 350-365.
%CITES:5302092154685965393
 
\item  Neel MC, {\bf Ross-Ibarra J}, Ellstrand NC (2001) Implications of mating patterns for conservation of the endangered plant \emph{Eriogonum ovalifolium} var. \emph{vineum}.  \textsc{American Journal of Botany} 88: 1214-1222.
%CITES:13724877679765824056
\end{etaremune}
 
\end{document}